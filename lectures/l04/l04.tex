\documentclass[10pt,aspectratio=169]{beamer}
\usepackage[utf8]{inputenc}
\usepackage{listings}
\usepackage{hyperref}
\usepackage{xcolor}
\usepackage{ragged2e}
\usepackage{graphicx}

\usetheme{Dresden}
%\usetheme{Bergen}

\title{Programming Techniques 2024-2025}
\author{Hannu Parviainen}
\institute{Universidad de la Laguna}
\date{\today}

\lstset{
  language=fortran,
  basicstyle=\ttfamily\scriptsize,
  backgroundcolor=\color{white},
  keywordstyle=\color{blue},
  commentstyle=\color{gray},
  stringstyle=\color{red},
  showstringspaces=false,
  breaklines=true,
  frame=lines
}

\subtitle{Lecture 4: Linked list exercise}

\begin{document}

\begin{frame}
  \titlepage
\end{frame}

\section{Recap - Variables}

\begin{frame}

Step 1: Write a module named \texttt{slist} saved in \texttt{slist.f90} that contains
\small
\begin{itemize}
    \item Type \texttt{Cell} consisting of an allocatable string ``\texttt{data}'' and a pointer to a \texttt{Cell} ``\texttt{next}''
    \item Type \texttt{SortedList} that contains an integer \texttt{length} and a pointer to a cell named \texttt{head}
    \item Subroutine \texttt{extend} that takes a \texttt{SortedList} and a string, creates a new \texttt{Cell}, adds it to the list so that the cells are sorted based on the string's first letter, and increments \texttt{length}.
    \item Subroutine \texttt{empty\_list} that takes a list and empties it taking care to deallocate memory correctly
    \item Function \texttt{pop} that returns the last cell of the list and removes it
    \item Recursive function \texttt{print\_forward} that prints all the strings in the list from last to first
    \item Recursive function \texttt{print\_reverse} that prints all strings in the list from first to last
\end{itemize}
\end{frame}

\begin{frame}

Step 2: Write a program in a separate file that uses the \texttt{SortedList} module, creates a sorted list, adds the strings:

\vspace{2mm}
river, spark, melody, whisper, canyon, drift, lantern, echo, quartz, and breeze

\vspace{2mm}
to it, and prints the list in ascending and descending order.

% Write a makefile that compiles the module and the program, and has additionally a target named “test” that compiles everything and runs the program

\end{frame}

\end{document}
