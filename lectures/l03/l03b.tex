\documentclass[10pt,aspectratio=169]{beamer}
\usepackage[utf8]{inputenc}
\usepackage{listings}
\usepackage{hyperref}
\usepackage{xcolor}
\usepackage{ragged2e}
\usepackage{graphicx}

\usetheme{Dresden}
%\usetheme{Bergen}

\title{Programming Techniques 2024-2025}
\author{Hannu Parviainen}
\institute{Universidad de la Laguna}
\date{\today}

\lstset{
  language=fortran,
  basicstyle=\ttfamily\scriptsize,
  backgroundcolor=\color{white},
  keywordstyle=\color{blue},
  commentstyle=\color{gray},
  stringstyle=\color{red},
  showstringspaces=false,
  breaklines=true,
  frame=lines
}

\subtitle{Lecture 3: More about data types}

\begin{document}

\begin{frame}
  \titlepage
\end{frame}

\section{Data type precision}
\begin{frame}
  \frametitle{Data type precision}
  \begin{itemize}
  \item   Fortran is used in very different computer architectures
  \begin{itemize}
    \item Can use different number of bytes
  \end{itemize}
  \end{itemize}

  \begin{itemize}
  \item   Many different ways to specify the precision of a data type
  \begin{itemize}
    \item C: float, double, int, longint, etc…
    \item Fortran: : integer(kind=x), real(kind=x)
  \end{itemize}
  where x is an integer stating the precision… but the the meaning of x depends on the compiler…
  \end{itemize}
\end{frame}

\begin{frame}[fragile]
  \frametitle{Precision in GNU Fortran}
  \begin{columns}[T]
    \begin{column}{0.5\textwidth}
      \begin{itemize}
        \item In GNU Fortran, `x` stands for the number of bytes used for the data type.
            \begin{itemize}
            \item x = 1, 2, 4, 8, 16
          \end{itemize}
        \item Defaults to:
          \begin{itemize}
            \item 4 for logical, integer, real, and complex.
            \item 8 for double precision.
            \item 1 for character.
          \end{itemize}
        \item But other compilers don't necessiraly use the same logic!
        \item Don’t use the kind specification directly if you want your program to be portable!
      \end{itemize}
    \end{column}

    \begin{column}{0.5\textwidth}
        In Gnu Fortran\\

      \begin{lstlisting}
real(kind=4) ! 32-bit single-precision float
real(kind=8) ! 64-bit double-precision float
integer(kind=4) ! 32-bit signed int
integer(kind=8) ! 64-bit signed int
      \end{lstlisting}
    \end{column}
  \end{columns}
\end{frame}


\begin{frame}[fragile]
  \frametitle{Portability with SELECTED\_KIND}
  \begin{columns}[T]
    \begin{column}{0.43\textwidth}
      \begin{itemize}
        \item For portability, use:
          \begin{itemize}
          \vspace*{2mm}
            \item \texttt{SELECTED\_INT\_KIND(R)} \\
            returns the kind value of the smallest integer type that can represent all values ranging from -10$^R$ to 10$^R$
            \vspace*{2mm}
            \item \texttt{SELECTED\_REAL\_KIND(P, R)} \\
            returns the kind value of a real data type with decimal precision of at least P digits and exponent range of at least R
          \end{itemize}
      \end{itemize}
    \end{column}

    \begin{column}{0.57\textwidth}
      \begin{lstlisting}
program ex3a
  implicit none
  integer, parameter :: si = selected_int_kind(5)
  integer, parameter :: li = selected_int_kind(15)
  integer(kind=si) :: o
  integer(kind=li) :: p
  print *, huge(o), huge(p)
end program ex3a
      \end{lstlisting}
    \end{column}
  \end{columns}
\end{frame}


\begin{frame}[fragile]
  \frametitle{Using ISO\_FORTRAN\_ENV}
  \begin{columns}[T]
    \begin{column}{0.5\textwidth}
      \begin{itemize}
      \item The Fortran 2003 standard includes an intrinsinc \texttt{ISO\_FORTRAN\_ENV} module that allows you to specify the number of bits directly.
        \item Does not necessarily guarantee the desired precision, but provides control over the number of bits.
        \item Common types:
          \begin{itemize}
            \item \texttt{int32}, \texttt{int64}
            \item \texttt{real32}, \texttt{real64}
          \end{itemize}
      \end{itemize}
    \end{column}

    \begin{column}{0.5\textwidth}
      \begin{lstlisting}
program ex3b
  use iso_fortran_env
  implicit none
  integer(int32) :: i
  integer(int64) :: j
  real(real32) :: x
  real(real64) :: y
  print *, huge(i), huge(j)
  print *, tiny(x), huge(x)
  print *, tiny(y), huge(y)
end program ex3b
      \end{lstlisting}
    \end{column}
  \end{columns}
\end{frame}


\begin{frame}[fragile]
  \frametitle{Using ISO\_C\_BIND}
  \begin{columns}[T]
    \begin{column}{0.5\textwidth}
      \begin{itemize}
        \item The Fortran 2003 standard includes an intrinsinc \texttt{ISO\_C\_BIND} module that enables interoperability with C.
        \item This allows direct relation to C data types.
        \item Common types:
          \begin{itemize}
            \item \texttt{c\_float}
            \item \texttt{c\_double}
          \end{itemize}
      \end{itemize}
    \end{column}

    \begin{column}{0.5\textwidth}
      \begin{lstlisting}
program ex3c
  use iso_c_bind
  implicit none
  real(c_float) :: x
  real(c_double) :: y
  print *, tiny(x), huge(x)
end program ex3c
      \end{lstlisting}
    \end{column}
  \end{columns}
\end{frame}

\section{Exercises}

\begin{frame}[fragile]
  \frametitle{Exercises}
  \begin{columns}[T]
    \begin{column}{0.5\textwidth}
      \textbf{Exercise 1:}
      \begin{itemize}
        \item Write a program that computes and prints the matrix multiplication of two real arrays.
      \end{itemize}
      \vspace*{0.2cm}
      \[
      A = \begin{pmatrix}
      3 & 2 & 4 & 1 \\
      2 & 4 & 2 & 2 \\
      1 & 2 & 3 & 7
      \end{pmatrix}
      \quad
      B = \begin{pmatrix}
      3 & 2 & 4 \\
      2 & 1 & 2 \\
      3 & 0 & 2
      \end{pmatrix}
      \]
    \end{column}

    \begin{column}{0.5\textwidth}
      \textbf{Exercise 2:}
      \begin{itemize}
        \item Write a program that reads two real arrays of length n and prints the sum of these arrays.
      \end{itemize}

      \textbf{Exercise 3:}
      \begin{itemize}
        \item Modify the matrix multiplication program to use a subroutine for the multiplication.
      \end{itemize}
    \end{column}
  \end{columns}
\end{frame}

\end{document}
