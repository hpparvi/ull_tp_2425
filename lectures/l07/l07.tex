\documentclass[10pt,aspectratio=169]{beamer}
\usepackage[utf8]{inputenc}
\usepackage{listings}
\usepackage{hyperref}
\usepackage{xcolor}
\usepackage{ragged2e}
\usepackage{graphicx}

\usetheme{Dresden}
%\usetheme{Bergen}

\title{Programming Techniques 2024-2025}
\author{Hannu Parviainen}
\institute{Universidad de la Laguna}
\date{\today}

\lstset{
  language=fortran,
  basicstyle=\ttfamily\scriptsize,
  backgroundcolor=\color{white},
  keywordstyle=\color{blue},
  commentstyle=\color{gray},
  stringstyle=\color{red},
  showstringspaces=false,
  breaklines=true,
  frame=lines
}

\subtitle{Lecture 7: Input/Output in Fortran: Read, Print, Write, and Formatting}

\begin{document}

\begin{frame}
  \titlepage
\end{frame}

\section{Introduction to I/O in Fortran}

\begin{frame}[fragile]
  \frametitle{Introduction to I/O in Fortran}
  \begin{columns}[T]
  \begin{column}{0.5\textwidth}
    Fortran provides three main IO statements
      \begin{itemize}
        \item \texttt{read}: Input
        \item \texttt{print}: Output to the console
        \item \texttt{write}: Output to files or console
      \end{itemize}

    \vspace*{2mm}
    IO formatting allows for better control over how input and output data is handled.

        \end{column}
    \begin{column}{0.5\textwidth}
\begin{lstlisting}
read *, v1, v2 ,v3, ...
read fmt, v1, v2 ,v3, ...
read (unit, fmt) v1, v2, v3, ...

print *, v1, v2, v3, ...
print fmt, v1, v2, v3, ...

write (unit, *) v1, v2, v3, ...
write (unit, fmt, advance, ...) v1, v2, v3, ...
\end{lstlisting}
    \end{column}
    \end{columns}
\end{frame}


\begin{frame}[fragile]
  \frametitle{Using the \texttt{read} Statement}
  The \texttt{read} statement is used for reading input from the user or files.

  \vspace*{2mm}
  Syntax:
      \begin{lstlisting}
read fmt, variable
read(unit, fmt) variable
      \end{lstlisting}
    \texttt{*} stands for default formatting and input and output (usually the terminal).

      \vspace*{2mm}
  Example:
  \begin{lstlisting}
program example_read
  implicit none
  integer :: a
  print *, "Enter an integer: "
  read(*, *) a
  print *, "You entered: ", a
end program example_read
  \end{lstlisting}
\end{frame}


\begin{frame}[fragile]
  \frametitle{Using the \texttt{print} Statement}
  The \texttt{print} statement is used for output to the console.

  \vspace*{2mm}
  Syntax:
      \begin{lstlisting}
print fmt, "Message"
      \end{lstlisting}

\vspace*{2mm}
  Example:
  \begin{lstlisting}
program example_print
  implicit none
  integer :: num = 10
  print *, "The value of num is: ", num
end program example_print
  \end{lstlisting}
\end{frame}


\begin{frame}[fragile]
  \frametitle{Using the \texttt{write} Statement}
  The \texttt{write} statement is more flexible than \texttt{print} and allows output to different IO units, like files or other devices than the console.

  \vspace*{2mm}
  Syntax:
      \begin{lstlisting}
write(unit, format) variable
      \end{lstlisting}

  \vspace*{2mm}
  Example:
  \begin{lstlisting}
program example_write
  implicit none
  integer :: i
  open(unit=10, file='output.txt')  ! Open a file
  do i = 1, 5
    write(10, *) "Line number: ", i
  end do
  close(10)  ! Close the file
end program example_write
  \end{lstlisting}
\end{frame}


\begin{frame}[fragile]
  \frametitle{IO Formatting in Fortran}
  Formatting is controlled with format descriptors.

    \begin{columns}[T]
    \begin{column}{0.5\textwidth}
  \begin{itemize}
    \item Format descriptors:
      \begin{itemize}
        \item \texttt{rIw}: integer.
        \item \texttt{rFw.d} floating point.
        \item \texttt{rEw.d} real in exponential notation.
        \item \texttt{rESw.d} real in scientific notation.
        \item \texttt{rAw} character string.
        \item \texttt{X} space.
        \item \texttt{/} blank line.
      \end{itemize}

    \item Modifiers:
      \begin{itemize}
        \item \texttt{r}: repeat count.
        \item \texttt{w}: field width.
        \item \texttt{d}: num. digits after the decimal point.
      \end{itemize}
  \end{itemize}
        \end{column}
    \begin{column}{0.5\textwidth}
  \begin{lstlisting}
program format_example
  implicit none
  integer :: a = 123
  real :: b = 456.789
  real, dimension(3) :: c = [1.2, 2.2, 1.2]
  print '(I5, 10X, F8.2, /, 3F8.1)', a, b, c
end program format_example
  \end{lstlisting}

  \begin{lstlisting}
     123            456.79
     1.2     2.2     1.2
  \end{lstlisting}
    \end{column}
    \end{columns}
\end{frame}


\section{Reading and Writing to Files}

\begin{frame}[fragile]
  \frametitle{Reading and Writing to Files}

      \begin{columns}[T]
    \begin{column}{0.55\textwidth}

    The \texttt{read} and \texttt{write} statements can be used to interact with files.
  \vspace*{2mm}

    Use the \texttt{open} statement to open a file and associate it with a unit number.

  \begin{itemize}
    \item \texttt{open(unit, file, status, action, ...)}
    \item \texttt{close(unit, ...)}
    \item \texttt{inquire(file, exists)}
  \end{itemize}

  \vspace*{2mm}
 \texttt{status} can be \texttt{new}, \texttt{old}, or \texttt{replace}.

   \vspace*{2mm}
  \texttt{action} can be \texttt{read}, \texttt{write}, or \texttt{readwrite}.
        \end{column}
    \begin{column}{0.45\textwidth}
  \begin{lstlisting}
program file_io
  implicit none
  integer :: i
  open(unit=20, file='data.txt', status='replace')
  do i = 1, 5
    write(20, '(I5)') i
  end do
  close(20)
end program file_io
  \end{lstlisting}
    \end{column}
    \end{columns}


\end{frame}


\end{document}
