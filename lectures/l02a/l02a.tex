\documentclass[10pt]{beamer}
\usepackage[utf8]{inputenc}
\usepackage{listings}
\usepackage{hyperref}
\usepackage{xcolor}

\usetheme{Dresden}

% Listings style for Fortran
\lstset{
  language=[90]Fortran,
  basicstyle=\ttfamily\footnotesize,
  backgroundcolor=\color{white},
  keywordstyle=\color{blue},
  commentstyle=\color{gray},
  stringstyle=\color{red},
  showstringspaces=false,
  breaklines=true,
}

\title{Programming Techniques 2024-2025}
\subtitle{Lecture 2: Recap of Basic Fortran Topics}
\author{Hannu Parviainen}
\institute{Universidad de la Laguna}
\date{\today}


\begin{document}

\begin{frame}
  \titlepage
\end{frame}

\section{Recap - Variables}

\begin{frame}[fragile]
  \frametitle{Recap - Variables}
  \begin{columns}[T]
    \begin{column}{0.5\textwidth}
    \begin{block}{Basic data types}
      \begin{itemize}
          \item logical
          \item integer
          \item real
          \item complex
          \item character
      \end{itemize}
    \end{block}

    \begin{block}{Variables}
      \begin{itemize}
        \item Are defined after `implicit none` but before the actual code.
        \item Can be initialised when defined.
      \end{itemize}        
    \end{block}
    \end{column}
    
    \begin{column}{0.5\textwidth}
      \begin{lstlisting}
program a
  implicit none
  integer :: j, i, k = 4
  real :: f, s = 0.12
  ! Some code here
end program a
      \end{lstlisting}
    \end{column}
  \end{columns}
\end{frame}

\section{Recap - Constants}

\begin{frame}[fragile]
  \frametitle{Recap - Constants}
  \begin{columns}[T]
    \begin{column}{0.45\textwidth}
      \begin{itemize}
        \item Constants are variables that do not change during the program's execution.
        \item Identified by the `parameter` modifier in the variable definition.
        \item Can be used to initialise other variables.
      \end{itemize}
    \end{column}

    \begin{column}{0.55\textwidth}
      \begin{lstlisting}
program a
  implicit none
  real, parameter :: pi = 3.14
  real, parameter :: two_pi = 2*pi
  integer :: j, i, k = 4
  real :: f, s = 0.12, r = two_pi
  ! Some code here
end program a
      \end{lstlisting}
    \end{column}
  \end{columns}
\end{frame}

\section{Recap - Conditional Execution}

\begin{frame}[fragile]
  \frametitle{Recap - Conditional Execution}
  \begin{columns}[T]
    \begin{column}{0.5\textwidth}
      \begin{lstlisting}
- if (x) a = 2

- if (x) then
    a = 2
  end if

- if (x) then
    a = 2
  else
    a = 3
  end if
      \end{lstlisting}
    \end{column}

    \begin{column}{0.5\textwidth}
      \begin{lstlisting}
- if (x) then
    a = 2
  else if (y) then
    a = 4
  else
    a = 3
  end if
      \end{lstlisting}
    \end{column}
  \end{columns}
\end{frame}

\section{Recap - Loops}

\begin{frame}[fragile]
  \frametitle{Recap - Loops}
  \begin{columns}[T]
    \begin{column}{0.5\textwidth}
    \begin{itemize}
        \item Loops allow you to execute a block of code multiple times.
        \item `do while`, and `do i =` loops are commonly used in Fortran.
      \end{itemize}
    \end{column}

    \begin{column}{0.5\textwidth}
      \begin{lstlisting}
do
  if (x) exit
  ! Some code
end do

do while (a < 10)
  ! Some code
end do

do i = 1, 10
  a = a + i
end do
      \end{lstlisting}
    \end{column}
  \end{columns}
\end{frame}


\end{document}
