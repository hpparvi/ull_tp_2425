\documentclass[10pt,aspectratio=169]{beamer}
\usepackage[utf8]{inputenc}
\usepackage{listings}
\usepackage{hyperref}
\usepackage{xcolor}
\usepackage{ragged2e}
\usepackage{graphicx}

\usetheme{Dresden}
%\usetheme{Bergen}

\title{Programming Techniques 2024-2025}
\author{Hannu Parviainen}
\institute{Universidad de la Laguna}
\date{\today}

\lstset{
  language=fortran,
  basicstyle=\ttfamily\scriptsize,
  backgroundcolor=\color{white},
  keywordstyle=\color{blue},
  commentstyle=\color{gray},
  stringstyle=\color{red},
  showstringspaces=false,
  breaklines=true,
  frame=lines
}

\subtitle{Lecture 2: Basics of Git and GitHub}

\begin{document}

\begin{frame}
  \titlepage
\end{frame}

\section{Introduction to Git}
\begin{frame}
  %\frametitle{What is Git?}
  \begin{block}{What is Git?}
    \textbf{Distributed Version Control System (VCS)} 
    \begin{itemize}
        \item Tracks changes in the source code and helps coordinate work among programmers.
        \item Designed to handle everything from small to very large projects quickly and efficiently. 
        \item Allows multiple developers to work on a project simultaneously without overwriting each other's changes. 
    \end{itemize}
  \end{block}

  \begin{block}{Distributed?}
  \justifying
  \begin{itemize}
      \item Each user has a full copy of the repository.
      \item Allows for a more flexible workflow than centralised VCSs.
      \item Branching is fast and does not require a connection to the VCS server.
  \end{itemize}
   \end{block}

\end{frame}


\begin{frame}
  \frametitle{Why Use Git?}
  \begin{itemize}
    \item \textbf{Collaboration:} Streamlines teamwork on projects.
    \item \textbf{Backup and Restore:} Safeguards code with version history.
    \item \textbf{Track History:} Allows you to see what changes were made and when.
    \item \textbf{Branching and Merging:} Supports multiple development lines.
  \end{itemize}
\end{frame}

\section{Getting Started with Git}

\begin{frame}
  \frametitle{Installing Git}
  \begin{block}{Windows}
    Install via \href{https://git-scm.com/download/win}{Git for Windows}.
  \end{block}
  \begin{block}{Mac}
    Install via Homebrew using `brew install git` or install \href{https://developer.apple.com/xcode/}{Xcode}.
  \end{block}
  \begin{block}{Linux}
    Install via package manager using `sudo apt-get install git` or equivalent.
  \end{block}
\end{frame}

\begin{frame}[fragile]
  \frametitle{Configuring Git}
  Set your username and email for Git commits:
  \begin{lstlisting}[language=bash]
$ git config --global user.name "Your Name"
$ git config --global user.email "you@example.com"
  \end{lstlisting}
  \begin{itemize}
    \item These settings are used to attribute commits to you.
    \item The `--global` flag applies settings for all repositories.
  \end{itemize}
\end{frame}

\section{Basic Git Commands}

\begin{frame}[fragile]
  \frametitle{Initializing a Repository}
  Initialize a new Git repository in the current directory:
  \begin{lstlisting}[language=bash]
$ git init
  \end{lstlisting}
  \begin{itemize}
    \item Creates a new `.git` subdirectory.
    \item Starts tracking versions for your project.
  \end{itemize}
\end{frame}

\begin{frame}[fragile]
  \frametitle{Checking Status}
  Check the status of your working directory and staging area:
  \begin{lstlisting}[language=bash]
$ git status
  \end{lstlisting}
  \begin{itemize}
    \item Shows tracked and untracked files.
    \item Indicates changes that are staged for commit.
  \end{itemize}
\end{frame}

% New slide: How Git Tracking Works
\begin{frame}
  \frametitle{How Git Tracking Works}
  \begin{itemize}
    \item \textbf{Working Directory:} Your local filesystem where you modify files.
    \item \textbf{Staging Area (Index):} A temporary area where you add changes to prepare for a commit.
    \item \textbf{Repository (History):} The database where commits are stored.
    \item \textbf{Tracking Changes:}
      \begin{itemize}
        \item Git monitors changes in tracked files.
        \item Untracked files are not monitored until added.
      \end{itemize}
    \item \textbf{Lifecycle of a File:}
      \begin{enumerate}
        \item Modify files in the working directory.
        \item Stage changes using `git add`.
        \item Commit changes to the repository with `git commit`.
      \end{enumerate}
  \end{itemize}
\end{frame}

\begin{frame}
  \frametitle{Staging vs. Committing Changes}
  \begin{itemize}
    \item \textbf{Staging Changes (\texttt{git add}):}
      \begin{itemize}
        \item Prepares selected changes for the next commit.
        \item Allows you to review and group changes.
      \end{itemize}
    \item \textbf{Committing Changes (\texttt{git commit}):}
      \begin{itemize}
        \item Records the staged changes into the repository history.
        \item Creates a new commit object with a unique ID.
        \item Includes a commit message describing the changes.
      \end{itemize}
    \item \textbf{Key Differences:}
      \begin{itemize}
        \item \textbf{Staging:} Prepares changes, but does not save them to history.
        \item \textbf{Committing:} Saves the staged changes permanently in the repository.
      \end{itemize}
  \end{itemize}
\end{frame}

\begin{frame}[fragile]
  \frametitle{Staging Changes with \texttt{git add}}
  Add files to the staging area:
  \begin{lstlisting}[language=bash]
$ git add filename
$ git add .
  \end{lstlisting}
  \begin{itemize}
    \item \texttt{git add filename}: Stages a specific file.
    \item \texttt{git add .}: Stages all changes in the current directory.
    \item \textbf{Staging:} Prepares changes to be included in the next commit.
  \end{itemize}
\end{frame}

\begin{frame}[fragile]
  \frametitle{Committing Changes with \texttt{git commit}}
  Commit the staged changes to the repository:
  \begin{lstlisting}[language=bash]
$ git commit -m "Commit message"
  \end{lstlisting}
  \begin{itemize}
    \item Records a snapshot of the staging area.
    \item \textbf{Commit Message:} Describes the changes made.
  \end{itemize}
\end{frame}

\section{Branching and Merging}

\begin{frame}
  \frametitle{Understanding Branches}
  \begin{itemize}
    \item \textbf{What is a Branch?}
      \begin{itemize}
        \item A parallel version of the repository.
        \item Allows you to work on different features independently.
      \end{itemize}
    \item \textbf{Why Use Branches?}
      \begin{itemize}
        \item Isolate development work without affecting the main codebase.
        \item Facilitate collaboration by allowing multiple features to be developed simultaneously.
      \end{itemize}
    \item \textbf{Default Branch:} Typically named `main` or `master`.
  \end{itemize}
\end{frame}

\begin{frame}[fragile]
  \frametitle{Creating and Switching Branches}
  Create a new branch and switch to it:
  \begin{lstlisting}[language=bash]
$ git branch new-feature
$ git checkout new-feature
  \end{lstlisting}
  Or combine both steps:
  \begin{lstlisting}[language=bash]
$ git checkout -b new-feature
  \end{lstlisting}
  \begin{itemize}
    \item \texttt{git branch new-feature}: Creates a new branch named `new-feature`.
    \item \texttt{git checkout new-feature}: Switches to the `new-feature` branch.
    \item \texttt{git checkout -b new-feature}: Creates and switches to `new-feature` in one command.
  \end{itemize}
\end{frame}

\begin{frame}[fragile]
  \frametitle{Merging Branches}
  Merge changes from `new-feature` branch into `main`:
  \begin{lstlisting}[language=bash]
$ git checkout main
$ git merge new-feature
  \end{lstlisting}
  \begin{itemize}
    \item \texttt{git checkout main}: Switches to the `main` branch.
    \item \texttt{git merge new-feature}: Merges `new-feature` into `main`.
    \item \textbf{Merging:} Combines changes from one branch into another.
  \end{itemize}
\end{frame}

\section{Introduction to GitHub}

\begin{frame}
  \frametitle{What is GitHub?}
  \begin{itemize}
    \item \textbf{Code Hosting Platform:} Hosts Git repositories online.
    \item \textbf{Facilitates Collaboration:} Tools for team communication and coordination.
    \item \textbf{Additional Features:}
      \begin{itemize}
        \item Issue tracking.
        \item Pull requests for code reviews.
        \item Wiki pages and documentation.
      \end{itemize}
  \end{itemize}
\end{frame}

\begin{frame}
  \frametitle{Setting Up GitHub}
  \begin{itemize}
    \item \textbf{Create an Account:} Sign up at \href{https://github.com}{github.com}.
    \item \textbf{Set Up SSH Keys:}
      \begin{itemize}
        \item Generate an SSH key pair on your local machine.
        \item Add the public key to your GitHub account.
      \end{itemize}
    \item \textbf{Configure Git to Use SSH:}
      \begin{itemize}
        \item Ensures secure communication with GitHub.
      \end{itemize}
  \end{itemize}
\end{frame}

\section{Working with GitHub}

\begin{frame}[fragile]
  \frametitle{Connecting a Local Repository to GitHub}
  Add a remote repository and push changes:
  \begin{lstlisting}[language=bash]
$ git remote add origin git@github.com:username/repo.git
$ git push -u origin main
  \end{lstlisting}
  \begin{itemize}
    \item \texttt{git remote add origin}: Links your local repo to GitHub.
    \item \texttt{git push -u origin main}: Pushes commits to the `main` branch on GitHub.
    \item \textbf{The \texttt{-u} Flag:} Sets `origin main` as the default upstream branch.
  \end{itemize}
\end{frame}

\begin{frame}
  \frametitle{Collaborating on GitHub}
  \begin{itemize}
    \item \textbf{Forking Repositories:} Create a personal copy of someone else's project.
    \item \textbf{Cloning Repositories:} Download a repository to your local machine.
    \item \textbf{Pull Requests:} Propose changes to a repository.
    \item \textbf{Code Reviews:} Collaboratively review code changes before merging.
    \item \textbf{Issue Tracking:} Report bugs or request features.
  \end{itemize}
\end{frame}

\section{Best Practices and Conclusion}

\begin{frame}
  \frametitle{Best Practices}
  \begin{itemize}
    \item \textbf{Write Meaningful Commit Messages:} Clearly describe what changes were made.
    \item \textbf{Keep Commits Small and Focused:} Easier to review and understand.
    \item \textbf{Regularly Pull Updates:} Keep your local repository up-to-date with the remote.
    \item \textbf{Use Branches for New Features:} Isolate development work.
    \item \textbf{Avoid Committing Sensitive Information:} Don't commit passwords or API keys.
  \end{itemize}
\end{frame}

\end{document}
