\documentclass[10pt,aspectratio=169]{beamer}
\usepackage[utf8]{inputenc}
\usepackage{listings}
\usepackage{hyperref}
\usepackage{xcolor}
\usepackage{ragged2e}
\usepackage{graphicx}

\usetheme{Dresden}
%\usetheme{Bergen}

\title{Programming Techniques 2024-2025}
\author{Hannu Parviainen}
\institute{Universidad de la Laguna}
\date{\today}

\lstset{
  language=fortran,
  basicstyle=\ttfamily\scriptsize,
  backgroundcolor=\color{white},
  keywordstyle=\color{blue},
  commentstyle=\color{gray},
  stringstyle=\color{red},
  showstringspaces=false,
  breaklines=true,
  frame=lines
}

\subtitle{Barnes-Hut Algorithm}

\begin{document}

\begin{frame}
  \titlepage
\end{frame}

\begin{frame}
  \frametitle{Problem: Direct simulation leads to O(N$^2$) scaling}
  \begin{itemize}
    \item We need to calculate the forces between all the particles in the simulation.
    \item The number of force calculations depends on the number of particles, \( N \):
    \begin{itemize}
      \item \( N = 2 \Rightarrow 1 \) calculation
      \item \( N = 3 \Rightarrow 3 \) calculations
      \item \( N = 4 \Rightarrow 6 \) calculations
      \item \( N = 10 \Rightarrow 45 \) calculations
      \item \( N = 100 \Rightarrow 4\,950 \) calculations
      \item \( N = 1000 \Rightarrow 499\,500 \) calculations
    \end{itemize}
    \item Computational complexity: $ \frac{1}{2}N(N-1) \sim N^2$
    \item Called $O(N^2)$ scaling.
  \end{itemize}
\end{frame}

\begin{frame}
  \frametitle{Computational Scaling}
  \begin{itemize}
    \item \( T_{N=100} / T_{N=10} = 100 \)
    \item \( T_{N=1000} / T_{N=100} = 100 \)
    \item \( T_{N=10\,000} / T_{N=1000} = 100 \)
    \item \( T_{N=10\,000} / T_{N=10} = 1\,000\,000 \)
    \item<2> \textbf{Not good!}
    \item<3> \textbf{Not good at all!}
  \end{itemize}
\end{frame}

\begin{frame}
  \frametitle{What can we do?}
  \begin{itemize}
    \item Find an algorithm that scales better.
    \item \textbf{Barnes-Hut Algorithm (Nature, V. 324 1986)}
    \item Scales as \( O(N \log N) \)
  \end{itemize}
  \begin{itemize}
    \item \( T_{N=100} / T_{N=10} = 20 \)
    \item \( T_{N=1000} / T_{N=100} = 15 \)
    \item \( T_{N=10\,000} / T_{N=1000} = 13 \)
    \item \( T_{N=10\,000} / T_{N=10} = 4000 \)
  \end{itemize}
\end{frame}

\begin{frame}
  \centering
  {\Large\textbf{How?}}
\end{frame}


\begin{frame}
  \centering
  {\Large\textbf{How?}}

  \vspace*{5mm}
    Approximate far-away clumps of particles as point masses.
\end{frame}

\begin{frame}
  \frametitle{Barnes-Hut}
    \centering
    \only<1>{\includegraphics[height=0.8\textheight]{bh01.pdf}}
    \only<2>{\includegraphics[height=0.8\textheight]{bh02.pdf}}
    \only<3>{\includegraphics[height=0.8\textheight]{bh03.pdf}}
    \only<4>{\includegraphics[height=0.8\textheight]{bh04.pdf}}
    \only<5>{\includegraphics[height=0.8\textheight]{bh05.pdf}}
\end{frame}

\begin{frame}
  \frametitle{How?}
  Use a spatial data structure to organise the particles.

    \vspace*{2mm}
  \begin{itemize}
    \item \textbf{Octree:}
    \begin{itemize}
     \item  A tree data structure that divides 3D space into hierarchical cubes.
     \item A 3D version of a quadtree.
    \end{itemize}

    \vspace*{2mm}
    \item \textbf{Octree node:}
    \begin{itemize}
    \item An octree node represents a cube in 3D space.
    \item Each node can be subdivided into eight smaller cubes (children nodes).
    \item A node has a link to its parent node.
    \item In our case, a node can contain:
      \begin{itemize}
        \item A single particle (either a pointer or an array index).
        \item Total mass of all the particles in the descendant nodes.
        \item 3D center of mass.
      \end{itemize}
    \end{itemize}

  \end{itemize}
\end{frame}


\begin{frame}[fragile]
  \frametitle{Octree Visualization}
  \centering
  \includegraphics[height=0.7\textheight]{octree2.pdf} \\
  \scriptsize Image by: \href{https://commons.wikimedia.org/wiki/File:Octree2.svg}{WhiteTimberwolf: CC BY-SA 3.0}
\end{frame}

\begin{frame}
  \frametitle{Octree Visualization}
    \centering
    \only<1>{\includegraphics[height=0.8\textheight]{octree_0.pdf}}
    \only<2>{\includegraphics[height=0.8\textheight]{octree_1.pdf}}
    \only<3>{\includegraphics[height=0.8\textheight]{octree_2.pdf}}
    \only<4>{\includegraphics[height=0.8\textheight]{octree_3.pdf}}
    \only<5>{\includegraphics[height=0.8\textheight]{octree_4.pdf}}
\end{frame}

\begin{frame}
  \frametitle{Octree Node Structure}
    \centering
  \includegraphics[height=0.8\textheight]{TP_2425_08_barnes_hutt_node.pdf}
\end{frame}

\begin{frame}
  \frametitle{Building an Octree}
  \begin{itemize}
    \item Find the node the particle belongs to.
    \item If the node is empty, assign the particle to the node.
    \item If the node already contains a particle:
      \begin{itemize}
        \item Subdivide the node.
        \item Move the existing particle to the correct child node.
        \item Try to assign the new particle to the correct child node.
        \item If the child node already has a particle, subdivide again.
      \end{itemize}
    \item Recursive subdivision makes this process clean and efficient.
  \end{itemize}
\end{frame}



\end{document}
